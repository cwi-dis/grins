\chapter{Audio file modules}

This chapter describes a set of modules that can be used to read audio
files, convert them to some desired format, and play them on the
system's audio device.  There are four main modules which are
described in more detail below.

The philosophy of the package is that an audio file is represented by
a reader object which has a set of methods to obtain information about
the data format and to obtain the audio data itself.  The format
information can be used to initialize the audio device, and the data
can then be written.  It may be that conversion is necessary before
the audio device can deal with the data.  This conversion takes place
when reading the data.

The following code fragment will play an audio file on the audio
device, converting it if needed.

\begin{verbatim}
dev = audiodev.writer()             # open output device
# open the audio file and convert to an acceptable format
rdr = audio.reader(file, dev.getformats(), dev.getframerates())
# tell device about format and frame rate
dev.setformat(rdr.getformat())
dev.setframerate(rdr.getframerate())
# read and play the file
while 1:
    data, nf = rdr.readframes(10000) # read and convert 10,000 audio frames
    if not data:                     # if no data, we reached the end
        break
    dev.writeframes(data)            # send data to device
dev.wait()                           # wait for output to drain
\end{verbatim}

Here is a short glossary of terms used in this document.

\begin{tableii}{|c|p{24em}|}{code}{Term}{Meaning}
\lineii{sample}{one audio sample}
\lineii{frame}{one audio sample for each channel}
\lineii{framerate}{the number of audio frames per second}
\end{tableii}

\section{Standard module audio}
\renewcommand{\indexsubitem}{(in module audio)}

This module defines only a single function:

\begin{funcdesc}{reader}{filename\optional{\, dstfmts\, rates}}
Create a reader object (see below) for audio file \var{filename}.  The
header of the file will be read, and the format and frame rate of the
audio file will be made available through the reader interface.  If
\var{dstfmts} or \var{rates} is specified, a converter is created that
converts the audio to the desired format and frame rate.  See the
\code{audioconvert} module for details.
\end{funcdesc}

This module also defines an exception:

\begin{excdesc}{Error}
Exception raised by the audio modules on various errors, such as bad
file format.
\end{excdesc}

\section{Standard module audiofile}
\renewcommand{\indexsubitem}{(in module audiofile)}

This module defines a single function:

\begin{funcdesc}{reader}{filename}
Create a reader object (see below) for audio file \var{filename}.  The
header of the file will be read, and the format and frame rate of the
audio file will be made available through the reader interface.
\end{funcdesc}

\section{Standard module audioconvert}
\renewcommand{\indexsubitem}{(in module audioconvert)}

This module defines a number of converters plus one function to create
a new reader object that does the necessary conversions to get from
one format to another.

\begin{funcdesc}{convert}{rdr\optional{\, dstfmts\, rates}}
Find a sequence of conversions to convert the audio from the given
reader object \var{rdr} to any of the formats given in \var{dstfmts}
at any of the frame rates given by \var{rates}.  If no \var{dstfmts}
is specified, or if it is \code{None}, the destination format does not
matter.  If no \var{rates} is specified, or if it is \code{None}, the
frame rate is unaltered.  Return a reader object that does the
requested conversion.

In finding a conversion, a sequence of conversions is constructed that
loses as little information as possible.  Information loss can occur
in two ways: the conversion causes fewer bits to represent the audio
(stereo to mono, 16 bit to 8 bit per sample), or the conversion causes
more bits to represent the audio (mono to stereo, $\mu$-law to 16-bit
PCM).  The latter type of loss is considered preferable to the former.
When frame rate conversion needs to take place, first a conversion is
found to a format that the frame rate conversion function can handle
(currently this means a linear format).  Then, if a destination format
is specified, a conversion is made to one of the destination formats.
\end{funcdesc}

\section{Standard module audioformat}
\renewcommand{\indexsubitem}{(in module audioformat)}

This module defines a number of objects that represent in-core audio
formats.  The objects have a number of methods with which more
information about the format can be got.  See below for a description
of these methods.  The frame rate is not part of the format, since it
would increase the number of possible formats to an unmanageable
number.

\begin{datadesc}{ulaw_mono}
8 bits per sample, 1 channel, $\mu$-law encoded.
\end{datadesc}

\begin{datadesc}{ulaw_stereo}
8 bits per sample, 2 channels interleaved (left channel first),
$\mu$-law encoded.
\end{datadesc}

\begin{datadesc}{dvi_mono}
4 bits per sample, 1 channel, DVI-ADPCM encoded (compatible with
Intel's DVI audio format).
\end{datadesc}

\begin{datadesc}{linear_8_mono_signed}
8 bits per sample, 1 channel, signed PCM encoded.  In this encoding,
sample values range from -128 to 127.
\end{datadesc}

\begin{datadesc}{linear_8_mono_excess}
8 bits per sample, 1 channel, excess-128 (unsigned) encoding.  In this
encoding, sample values range from 0 to 255 with 128 being the
mid-point.
\end{datadesc}

\begin{datadesc}{linear_16_mono_big}
16 bits per sample, 1 channel, big-endian linear encoding.  Two bytes
per sample, with the high-order byte coming first.
\end{datadesc}

\begin{datadesc}{linear_16_mono_little}
16 bits per sample, 1 channel, little-endian linear encoding.  Two
bytes per sample, with the low-order byte coming first.
\end{datadesc}

\begin{datadesc}{linear_16_mono}
Either \code{linear_16_mono_big} or \code{linear_16_mono_little},
depending on the natural byte-order of the system.
\end{datadesc}

\begin{datadesc}{linear_8_stereo_signed}
8 bits per sample, 2 channels interleaved (left channel first), signed
PCM encoded.  In this encoding, sample values range from -128 to 127.
\end{datadesc}

\begin{datadesc}{linear_8_stereo_excess}
8 bits per sample, 2 channels interleaved (left channel first),
excess-128 (unsigned) encoding.  In this encoding, sample values range
from 0 to 255 with 128 being the mid-point.
\end{datadesc}

\begin{datadesc}{linear_16_stereo_big}
16 bits per sample, 2 channels interleaved (left channel first),
big-endian linear encoding.  Two bytes per sample, with the high-order
byte coming first.
\end{datadesc}

\begin{datadesc}{linear_16_stereo_little}
16 bits per sample, 2 channels interleaved (left channel first),
little-endian linear encoding.  Two bytes per sample, with the
low-order byte coming first.
\end{datadesc}

\begin{datadesc}{linear_16_stereo}
Either \code{linear_16_stereo_big} or \code{linear_16_stereo_little},
depending on the natural byte-order of the system.
\end{datadesc}

Format objects have the following methods:
\renewcommand{\indexsubitem}{(audioformat object methods)}

\begin{funcdesc}{getname}{}
Returns the name of the format.
\end{funcdesc}

\begin{funcdesc}{getdescr}{}
Returns a short (human-readable) description of the format.
\end{funcdesc}

\begin{funcdesc}{getnchannels}{}
Returns the number of channels of the format.
\end{funcdesc}

\begin{funcdesc}{getencoding}{}
Returns a string describing the type of encoding (u-law,
linear-signed, etc.).
\end{funcdesc}

\begin{funcdesc}{getblocksize}{}
Returns the size in bytes of the smallest unit.  Fewer bytes would
result in an incomplete frame.
\end{funcdesc}

\begin{funcdesc}{getfpb}{}
Returns the number of frames in a single block.  Returns 0 if the
number of frames per block is variable.
\end{funcdesc}

The linear formats also have the following method:

\begin{funcdesc}{getbps}{}
Returns the number of bits per sample.
\end{funcdesc}

\section{Standard module audiomerge}
\renewcommand{\indexsubitem}{(in module audiomerge)}

This module defines a single function:

\begin{funcdesc}{merge}{\optional{reader...}}
Create a merge object that represents a set of merged audio streams.
\code{merge} may be called with any number of arguments.  The
arguments should either be a reader object instance, or a tuple
consisting of a reader object instance and a callback as described
below in the \code{add} method.  If any arguments are given, the
\code{add} method is called for each of the arguments.
\end{funcdesc}

The merge object has all of the reader object methods, except that
\code{getpos} and \code{setpos} currently raise an exception and
should not be used.  The other methods should only be called when at
least one stream is added.  In addition, a merge object has the
following methods:
\renewcommand{\indexsubitem}{(audiomerge object methods)}

\begin{funcdesc}{add}{rdr\optional{\, callback}}
Add the \var{rdr} object to the list of merged audio streams.  If
\var{callback} is given, it is a callback that is called when the end
of the audio stream is reached.  The callback is a tuple, of which the
first element is a callable object, and the second a tuple giving the
arguments with which the callable object should be called.

The \var{rdr} may have any format, but conversion to some internal
format may be added by \code{add}.

The \code{add} method may be called at any time.
\end{funcdesc}

\begin{funcdesc}{delete}{rdr}
Remove the given \var{rdr} from the list of merged readers.
\end{funcdesc}

\section{Reader object}
\renewcommand{\indexsubitem}{(reader object methods)}

A reader object is returned by \code{audio.reader()}, by
\code{audiofile.reader()}, and by \code{audioconvert.convert()}.
Reader objects provide a uniform interface to the audio data and
format descriptions.  Reader objects have the following methods:

\begin{funcdesc}{getformat}{}
Returns the format object that describes the audio data.
\end{funcdesc}

\begin{funcdesc}{getframerate}{}
Returns the frame rate of the audio data.
\end{funcdesc}

\begin{funcdesc}{readframes}{\optional{nframes}}
Returns a tuple consisting of the audio data of up to \var{nframes}
frames, and the number of frames actually read.  If \var{nframes} is
omitted, or is negative, returns all remaining audio data.
\end{funcdesc}

\begin{funcdesc}{getnframes}{}
Returns the number of remaining audio frames in the data.  Returns -1
if the number of frames is unknown.
\end{funcdesc}

\begin{funcdesc}{rewind}{}
Resets the audio stream to the beginning.
\end{funcdesc}

\begin{funcdesc}{getpos}{}
Returns a magic cookie representing the current position in the audio
stream.  The only use of this cookie is to pass it to the
\code{setpos} method.
\end{funcdesc}

\begin{funcdesc}{setpos}{pos}
Set the audio stream to the position \var{pos}.  Only values for
\var{pos} as returned by \code{getpos} are valid.
\end{funcdesc}

\section{Standard module audiodev}
\renewcommand{\indexsubitem}{(in module audiodev)}

This module is a system-independent interface to the audio hardware.
The way to use this module is to first obtain an audio device object.
Before data can be written, the format and frame rate parameters have
to be set.  When all data has been written, either \code{wait} or
\code{stop} should be called.  These methods may reset the audio
hardware to the initial state.

There is only one function in this module:

\begin{funcdesc}{writer}{\optional{fmt\, qsize}}
Returns an object representing an audio output stream.  \var{qsize} is
the queue size in bytes of the internal output buffer.  Whether or not
this argument is used is system-dependent.

If a \var{fmt} argument is passed, it is passed to the \code{setformat}
method (see below).
\end{funcdesc}

The audio device object has the following methods:
\renewcommand{\indexsubitem}{(audio device output object methods)}

\begin{funcdesc}{getformats}{}
Returns a tuple containing the audio formats that this device can
handle.
\end{funcdesc}

\begin{funcdesc}{getframerates}{}
Returns a sequence object (tuple, list, xrange object) of frame rates
that this device can handle.  Returns \code{None} if the device can
handle any output frame rate.
\end{funcdesc}

\begin{funcdesc}{setformat}{fmt}
Set the output format.  This method must be called before
\code{writeframes} is called, and the format given must be one of the
formats returned by \code{getformats}.
\end{funcdesc}

\begin{funcdesc}{setframerate}{rate}
Set the output frame rate.  This method must be called before
\code{writeframes} is called, and, if \code{getrates} does not return
\code{None}, the frame rate given must be one of the rates returned by
\code{getrates}
\end{funcdesc}

\begin{funcdesc}{writeframes}{data}
Send the audio \var{data} to the audio device.  This method may or may
not block, and it may or may not return before all data has been sent.
\end{funcdesc}

\begin{funcdesc}{wait}{}
Wait until all audio data that is still in the output queue has been
played.
\end{funcdesc}

\begin{funcdesc}{stop}{}
Stop playing audio immediately.
\end{funcdesc}

\begin{funcdesc}{getfilled}{}
Returns the number of frames still in the output queue of the audio
device.
\end{funcdesc}

\begin{funcdesc}{getfillable}{}
Return the number of frames that can be written without
\code{writeframes} blocking.
\end{funcdesc}
